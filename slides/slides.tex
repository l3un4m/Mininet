\documentclass{beamer}
\usetheme{metropolis}
\usepackage{enumitem}

\graphicspath{{media/}}
\usepackage{booktabs} % Required for better table rules
\usepackage{multicol}
%Information to be included in the title page:
\setbeamertemplate{frame footer}{\insertdate{} - \insertshortauthor{}}
\title{Mininet}
\author{Manuel Dias}
\institute{UCLouvain}
\date{16/05/2025}

\begin{document}

\frame{\titlepage}


\begin{frame}
\frametitle{Dns Reflection}
\begin{columns}
    % Column 1
    \begin{column}{0.5\textwidth}
        \Large Attack
        \begin{figure}
            \centering
            \includegraphics[width=\textwidth]{dns_attack.jpg}\\
        \end{figure}
            \begin{itemize}[label={}]
                \item \footnotesize Send Dns Requests in the name of the victim
                \item \footnotesize DNS Server processes and sends response to the victim
                \item \footnotesize Victim is overloaded with Traffic becoming Unavailable
            \end{itemize}
    \end{column}
    % Column 2
    \begin{column}{0.5\textwidth}
        \Large Defense
        \begin{figure}
            \centering
            \includegraphics[width=\textwidth]{dns_defense.jpg}\\
        \end{figure}
            \begin{itemize}[label={}]
                \item \footnotesize Send Dns Requests in the name of the victim
                \item \footnotesize DNS Server doesn't recognize the source of the request meaning it's spoofed so it
                \item \footnotesize Drops the packet
            \end{itemize}
    \end{column}
\end{columns}
\end{frame}

\begin{frame}
\frametitle{Arp Poisoning}
\begin{columns}
    % Column 1
    \begin{column}{0.5\textwidth}
        \Large Attack
        \begin{figure}
            \centering
            \includegraphics[width=0.8\textwidth]{arp_attack.jpg}\\
        \end{figure}
            \begin{itemize}[label={}]
                \item \footnotesize Send Spoofed ARP packet saying that our MAC Address matches the Default Gateway's
                \item \footnotesize Victim Updates it's ARP Table with wrong MAC Address
               \item \footnotesize Attacker receiver all the traffic coming out of Victim
            \end{itemize}
    \end{column}
    % Column 2
    \begin{column}{0.5\textwidth}
        \Large Defense
        \begin{figure}
            \centering
            \includegraphics[width=0.8\textwidth]{arp_defense.jpg}\\
        \end{figure}
            \begin{itemize}[label={}]
                \item \footnotesize Send Spoofed ARP packet saying that our MAC Address matches the Default Gateway's
                \item \footnotesize Victim checks it's ARP Table for changes in the MAC
                \item \footnotesize Ignores the ARP Change request
            \end{itemize}
    \end{column}
\end{columns}
\end{frame}

\begin{frame}
\frametitle{Network Scan}
\begin{columns}
    % Column 1
    \begin{column}{0.5\textwidth}
        \Large Attack
        \begin{figure}
            \centering
            \includegraphics[width=0.8\textwidth]{scan_attack.jpg}\\
        \end{figure}
            \begin{itemize}[label={}]
                \item \footnotesize Send ICMP Requests to all the addresses of a given sub-domain
                \item \footnotesize Send TCP SYN packets to all of the found Hosts that responded
               \item \footnotesize  Save the addresses and their open ports
            \end{itemize}
    \end{column}
    % Column 2
    \begin{column}{0.5\textwidth}
        \Large Defense
        \begin{figure}
            \centering
            \includegraphics[width=0.8\textwidth]{scan_defense.jpg}\\
        \end{figure}
            \begin{itemize}[label={}]
                \item \footnotesize Send ICMP Requests/TCP SYN Packets to all the addresses of a given sub-domain
                \item \footnotesize Firewall sees that the amount of these type of packets that were received surpasses the threshold so it drops
            \end{itemize}
    \end{column}
\end{columns}
\end{frame}
\end{document}
